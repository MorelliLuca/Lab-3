\documentclass[]{article}
\usepackage{multirow}
\usepackage{amsmath}
\usepackage{circuitikz}
%opening
\title{Misura della caratteristica I-V di due diodi a giunzione p-n}
\author{Cristina Caprioglio, Luca Morelli}
\date{Primo turno, tavolo 3}

\begin{document}

\maketitle

\begin{abstract}
Lo mettiamo? Lei non lo menziona
\end{abstract}

\section{Scopo della prova}
La prova consisteva nella misura delle caratteristiche I-V di due diodi a giunzione p-n, uno al silicio e uno al germanio. Abbiamo inoltre realizzato dei fit su ROOT in modo da ricavare i parametri fisici corrente inversa ``$  I_{0}$" e ``$\eta V_{T}$", rispettivamente la corrente inversa e il prodotto tra il fattore di idealità e l'equivalente della temperatura in volt. 
\section{Procedura}
\begin{center}
\begin{circuitikz}
	\draw
	(0,4) node[above]{$Ground$} to[short, *-]
	(0,2) to[potentiometer, -, l=1$ k\Omega $] (2,2) 
	(2.5,0) node[below]{$B$} to[empty diode, *-*] (1.25,0)node[below]{$A$}--(0,0)--(0,2)
	(2,4) node[above]{$+5V$} to[short, *-] (2,2)
	;
	\draw [<-]
	(1,1.5)--(1,1)--(2,1);
	\draw
	(2,1)--(2.5,1) node[above]{$C$} to[short, *-] (3,1)
	 to[short, -*] (3, 3.5)node[above]{$(Red)$} (3.5, 4)node[above]{$Multimeter$}
	;
	\draw
	(4,3.5)node[above]{$(Black)$} to[short, *-] (4,0)--(3.5,0)node[below]{$D$}to[short, *-](2.5,0)
	;
\end{circuitikz}
\end{center}
Per prima cosa abbiamo eseguito la calibrazione della tensione misurata con l'oscilloscopio, mettendola in relazione con quella data dal multimetro. Per fare ciò abbiamo collegato l'oscilloscopio al punto C e abbiamo cortocircuitato i punti A-B e abbiamo preso 10 misure tra i 50 e i 760 mV. Abbiamo prima preso il valore dell'oscilloscopio e poi quello del multimetro.
Spostando poi il potenziometro fuori dal circuito abbiamo regolato la resistenza a $ 500 \,\Omega $, per poi reinserirlo e mettere anche tra i punti A e B il diodo, prima al silicio e poi al germanio, con il catodo nel punto A. Dopo aver spostato l'oscilloscopio nel punto D abbiamo effettuato 16 misure per il silicio e 23 per il germanio, agendo sul potenziometro per variare la tensione e leggendo poi la corrente dal multimetro. Infine, abbiamo riportato i dati su dei grafici con scala semi-logaritmica ed eseguito i fit per ottenere i parametri ricercati.
\section{Materiali utilizzati}
\begin{itemize}
	\item Potenziometro da $ 1 \,k\Omega $
	\item Diodo p-n: AAZ15/OA47 Germanio
	\item Diodo p-n: 1N914A/1N4446/1N4148 Silicio
	\item Cavetti
	\item Cacciavite
	\item Cavi a doppia banana
	\item Breadboard
\end{itemize}
\section{Strumentazione}
\begin{itemize}
	\item Alimentatore a bassa tensione
	\item Oscilloscopio ISO-TECH, ISR 622 20MHz
	\item Multimetro digitale
\end{itemize}
\section{Misurazioni}
La tabella di seguito riporta i valori relativi a fondo scala, risoluzione e precisione dei vari strumenti:
\begin{center}
\begin{tabular}{|c|c|c|c|}
	\cline{2-4}
	\multicolumn{1}{c|}{} & Fondo scala & Risoluzione & Precisione \\
	\hline
	\multirow{4}{*}{Oscilloscopio (mV)} & 10 & 2 & 3\% \\
	\cline{2-4}
	& 50 & 10 & 3\% \\
	\cline{2-4}
	& 100 & 20 & 3\% \\
	\cline{2-4}
	& 200 & 40 & 3\% \\
	\hline
	\multirow{4}{*}{Multimetro (mV)} & 10 & 2 & 3\% \\
	\cline{2-4}
	& 50 & 10 & 3\% \\
	\cline{2-4}
	& 100 & 20 & 3\% \\
	\cline{2-4}
	& 200 & 40 & 3\% \\
	\hline
	\multirow{4}{*}{Multimetro (mA)} & 10 & 2 & 3\% \\
	\cline{2-4}
	& 50 & 10 & 3\% \\
	\cline{2-4}
	& 100 & 20 & 3\% \\
	\cline{2-4}
	& 200 & 40 & 3\% \\
	\hline
\end{tabular}
\end{center}
Per il calcolo degli errori relativi alle misure effettuate con l'oscilloscopio si è usata la seguente formula:
\begin{equation}
	\sigma=\sqrt{(\sigma_{L})^{2}+(\sigma_{Z})^{2}+(\sigma_{C})^{2}}
\end{equation}
$ 	\sigma_{C}= (misura*0.03) $ è l'errore del costruttore.
\begin{equation*}
	\sigma_{L}=\sigma_{Z}=\frac{fondo \:scala}{5}*\#tacchette \:apprezzabili
\end{equation*}
$ \sigma_{Z} $ è l'errore sullo zero, il fondo scala vale 10 mV e il numero di tacchette apprezzabili 5.\\
$ \sigma_{L} $ è l'errore sulla lettura e il fondo scala varia in base alla misura, mentre il numero di tacchette apprezzabili é stato considerato 5 per tutte le misure con eccezion fatta per quelle relative a 550, 570 e 620 mV nella misura della caratteristica del silicio, dove ne abbiamo considerate 10.
\subsection{Calibrazione dell'oscilloscopio}
\begin{center}
	\begin{tabular}{|c|c|c|c|}
		\hline
		Tensione oscilloscopio (mV)& Fondo scala (mV) & Tensione multimetro (mV) \\
		\hline
		$ 50\pm 14 $ &$ 10 $ & $ 48.20\pm 0.34 $ \\
		\hline
		$ 130\pm51 $ &$ 50 $ & $ 123.40\pm 0.57 $ \\
		\hline
		$ 210\pm 51 $ &$ 50 $ & $ 202.6\pm 0.81 $ \\
		\hline
		$ 280\pm 101 $ &$ 100 $ & $ 268.8\pm 1 $ \\
		\hline
		$ 360\pm 101 $ &$ 100 $ & $ 349.3\pm 1.2 $ \\
		\hline
		$ 440\pm 101 $ &$ 100 $ & $ 428\pm 2.4 $ \\
		\hline
		$ 520\pm 102 $ &$ 100 $ & $ 505\pm 2.5 $ \\
		\hline
		$ 600\pm 201 $ &$ 200 $ & $ 571\pm 2.6 $ \\
		\hline
		$ 680\pm 201 $ &$ 200 $ & $ 654\pm 2.7 $ \\
		\hline
		$ 760\pm 202 $ &$ 200 $ & $ 734\pm 2.7 $ \\
		\hline
	\end{tabular}
\end{center}

\subsection{Silicio}
\begin{center}
	\begin{tabular}{|c|c|c|c|}
		\hline
		Tensione oscilloscopio (mV)& Fondo scala (mV) & Corrente multimetro (mA) \\
		\hline
		$ 420\pm 101 $ &$ 100 $ & $ 0.016\pm 0.002 $ \\
		\hline
		$440\pm101 $ &$ 100 $ & $ 0.025\pm0.002 $ \\
		\hline
		$ 460\pm 101 $ &$ 100 $ & $ 0.038\pm 0.002 $ \\
		\hline
		$ 500\pm 101 $ &$ 100 $ & $ 0.082\pm 0.002 $ \\
		\hline
		$ 520\pm 102 $ &$ 100 $ & $0.121\pm 0.002$ \\
		\hline
		$ 540\pm 102 $ &$ 100 $ & $ 0.185\pm 0.003 $ \\
		\hline
		$ 550\pm 202 $ &$ 100 $ & $ 0.213\pm 0.003 $ \\
		\hline
		$ 560\pm 102 $ &$ 100 $ & $ 0.284\pm 0.003 $ \\
		\hline
		$ 570\pm 202 $ &$ 100 $ & $ 0.297\pm 0.003 $ \\
		\hline
		$ 580\pm 102 $ &$ 100 $ & $ 0.350\pm 0.004 $ \\
		\hline
		$ 600\pm 201 $ &$ 200 $ & $ 0.602\pm0.004 $ \\
		\hline
		$ 620\pm 401 $ &$ 200 $ & $ 0.738\pm0.005 $ \\
		\hline
		$ 640\pm 201 $ &$ 200 $ & $ 1.207\pm0.007 $ \\
		\hline
		$ 680\pm 201 $ &$ 200 $ & $ 2.238\pm 0.010 $ \\
		\hline
		$ 720\pm 201 $ &$ 200 $ & $ 2.615\pm 0.012 $ \\
		\hline
		$ 760\pm 202 $ &$ 200 $ & $ 3.701\pm 0.017 $ \\
		\hline
			\end{tabular}
	\end{center}
\subsection{Germanio}
\begin{center}
	\begin{tabular}{|c|c|c|c|}
		\hline
		Tensione oscilloscopio (mV)& Fondo scala (mV) & Corrente multimetro (mA) \\
		\hline
		$ 70\pm 51 $ &$ 50 $ & $ 0.014\pm 0.002 $ \\
		\hline
		$ 80\pm 51 $ &$ 50 $ & $ 0.020\pm 0.002 $ \\
		\hline
		$ 90\pm 51 $ &$ 50 $ & $ 0.026\pm 0.002 $ \\
		\hline
		$ 100\pm 51 $ &$ 50 $ & $ 0.034\pm 0.002 $ \\
		\hline
		$110\pm 51 $ &$ 50 $ & $ 0.045\pm 0.002 $ \\
		\hline
		$ 120\pm 51 $ &$ 50 $ & $ 0.056\pm 0.002 $ \\
		\hline
		$ 130\pm 51 $ &$ 50 $ & $ 0.071\pm 0.002 $ \\
		\hline
		$ 140\pm 51 $ &$ 50 $ & $ 0.089\pm 0.002 $ \\
		\hline
		$ 150\pm 51 $ &$ 50 $ & $ 0.109\pm 0.002 $ \\
		\hline
		$ 160\pm 51 $ &$ 50 $ & $ 0.134\pm 0.003 $ \\
		\hline
		$ 170\pm 51 $ &$ 50 $ & $ 0.162\pm 0.003 $ \\
		\hline
		$ 180\pm 51 $ &$ 50 $ & $ 0.200\pm 0.003 $ \\
		\hline
		$ 190\pm 51 $ &$ 50 $ & $ 0.244\pm 0.003 $ \\
		\hline
		$ 200\pm 51 $ &$ 50 $ & $ 0.305\pm 0.003 $ \\
		\hline
		$ 210\pm 51 $ &$ 50 $ & $ 0.323\pm 0.003 $ \\
		\hline
		$ 220\pm 51 $ &$ 50 $ & $ 0.441\pm 0.004 $ \\
		\hline
		$ 230\pm 51 $ &$ 50 $ & $ 0.451\pm 0.004 $ \\
		\hline
		$ 240\pm 52 $ &$ 50 $ & $ 0.537\pm 0.004 $ \\
		\hline
		$ 250\pm 52 $ &$ 50 $ & $ 0.712\pm 0.005 $ \\
		\hline
		$ 260\pm 52 $ &$ 50 $ & $ 0.730\pm 0.005 $ \\
		\hline
		$ 270\pm 52 $ &$ 50 $ & $ 0.850\pm 0.005 $ \\
		\hline
		$ 280\pm 52 $ &$ 50 $ & $ 0.990\pm 0.006 $ \\
		\hline
		$ 290\pm 52 $ &$ 50 $ & $ 1.118\pm 0.006 $ \\
		\hline
	\end{tabular}
\end{center}
\section{Grafici}

\subsection{Calibrazione dell'oscilloscopio}
\subsection{Silicio}
\subsection{Germanio}
\section*{Conclusioni}
\end{document}
