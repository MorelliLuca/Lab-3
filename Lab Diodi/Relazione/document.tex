\documentclass[]{article}

%opening
\title{Misura della caratteristica I-V di due diodi a giunzione p-n}
\author{Cristina Caprioglio, Luca Morelli}
\date{Primo turno, tavolo 3}

\begin{document}

\maketitle

\begin{abstract}
Lo mettiamo? Lei non lo menziona
\end{abstract}

\section{Scopo della prova}
La prova consisteva nella misura delle caratteristiche I-V di due diodi a giunzione p-n, uno al silicio e uno al germanio. Si sono inoltre realizzati dei fit su ROOT in modo da ricavare i parametri fisici corrente inversa ``$  I_{0}$" e ``$\eta V_{T}$", rispettivamente la corrente inversa e il prodotto tra il fattore di idealità e l'equivalente della temperatura in volt. 
\section{Procedura}
%teoricamente non la richiede, ma io metterei qui lo schema del circuito invece di fare una sezione intera solo con quello
\section{Materiali utilizzati}
\section{Strumentazione}
\section{Misurazioni}
\subsection{Calibrazione dell'oscilloscopio}
\subsection{Silicio}
\subsection{Germanio}
\section{Grafici}
\subsection{Calibrazione dell'oscilloscopio}
\subsection{Silicio}
\subsection{Germanio}
\section*{Conclusioni}
\end{document}
