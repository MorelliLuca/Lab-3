\documentclass[]{article}

%opening
\title{Misura della caratteristica I-V di due diodi a giunzione p-n}
\author{Cristina Caprioglio, Luca Morelli}
\date{Primo turno, tavolo 3}

\begin{document}

\maketitle

\begin{abstract}
Lo mettiamo? Lei non lo menziona
\end{abstract}

\section{Scopo della prova}
La prova consisteva nella misura delle caratteristiche I-V di due diodi a giunzione p-n, uno al silicio e uno al germanio. Abbiamo inoltre realizzato dei fit su ROOT in modo da ricavare i parametri fisici corrente inversa ``$  I_{0}$" e ``$\eta V_{T}$", rispettivamente la corrente inversa e il prodotto tra il fattore di idealità e l'equivalente della temperatura in volt. 
\section{Procedura}
%teoricamente non la richiede, ma io metterei qui lo schema del circuito invece di fare una sezione intera solo con quello
Per prima cosa abbiamo eseguito la calibrazione della tensione misurata con l'oscilloscopio, mettendola in relazione con quella data dal multimetro. Per fare ciò abbiamo collegato l'oscilloscopio al punto C e abbiamo cortocircuitato i punti A-B e abbiamo preso 10 misure tra i 50 e i 760 mV. Abbiamo prima preso il valore dell'oscilloscopio e poi quello del multimetro.
Spostando poi il potenziometro fuori dal circuito abbiamo regolato la resistenza a $ 500 \,\Omega $, per poi reinserirlo e mettere anche tra i punti A e B il diodo, prima al silicio e poi al germanio, con il catodo nel punto A. Dopo aver spostato l'oscilloscopio nel punto D abbiamo effettuato 16 misure per il silicio e 23 per il germanio, agendo sul potenziometro per variare la tensione e leggendo poi la corrente dal multimetro. Infine, abbiamo riportato i dati su dei grafici con scala semi-logaritmica ed eseguito i fit per ottenere i parametri ricercati.
\section{Materiali utilizzati}
\begin{itemize}
	\item Potenziometro da $ 1 \,k\Omega $
	\item Diodo p-n: AAZ15/OA47 Germanio
	\item Diodo p-n: 1N914A/1N4446/1N4148 Silicio
	\item Cavetti
	\item Cacciavite
	\item Cavi a doppia banana
	\item Breadboard
\end{itemize}
\section{Strumentazione}
\begin{itemize}
	\item Alimentatore a bassa tensione
	\item Oscilloscopio ISO-TECH, ISR 622 20MHz
	\item Multimetro digitale
\end{itemize}
\section{Misurazioni}
\subsection{Calibrazione dell'oscilloscopio}
\begin{center}
	\begin{tabular}{|c|c|c|c|}
		\hline
		Tensione oscilloscopio (mV)& Fondo scala (mV) & Tensione multimetro (mV) \\
		\hline
		$ 50\pm 14 $ &$ 10 $ & $ 48.20\pm 0.34 $ \\
		\hline
		$ 130\pm51 $ &$ 50 $ & $ 123.40\pm 0.57 $ \\
		\hline
		$ 210\pm 51 $ &$ 50 $ & $ 202.6\pm 0.81 $ \\
		\hline
		$ 280\pm 101 $ &$ 100 $ & $ 268.8\pm 1 $ \\
		\hline
		$ 360\pm 101 $ &$ 100 $ & $ 349.3\pm 1.2 $ \\
		\hline
		$ 440\pm 101 $ &$ 100 $ & $ 428\pm 2.4 $ \\
		\hline
		$ 520\pm 102 $ &$ 100 $ & $ 505\pm 2.5 $ \\
		\hline
		$ 600\pm 201 $ &$ 200 $ & $ 571\pm 2.6 $ \\
		\hline
		$ 680\pm 201 $ &$ 200 $ & $ 654\pm 2.7 $ \\
		\hline
		$ 760\pm 202 $ &$ 200 $ & $ 734\pm 2.7 $ \\
		\hline
	\end{tabular}
\end{center}
\subsection{Silicio}
\subsection{Germanio}
\section{Grafici}
\subsection{Calibrazione dell'oscilloscopio}
\subsection{Silicio}
\subsection{Germanio}
\section*{Conclusioni}
\end{document}
